\documentclass[a4paper, 11pt]{article}
\usepackage[left=2cm,top=3cm,text ={17cm,24cm}]{geometry}
\usepackage[czech]{babel}
\usepackage{czech}
\usepackage[latin2]{inputenc}
\usepackage{times}
\bibliographystyle{czechiso}
\begin{document}
\begin{center} 
\pagenumbering{gobble}
\Huge
\textsc{Vysoké učení
technické v Brně\\ \huge Fakulta informačních technologií}\\
\vspace{\stretch{0.382}}

\LARGE Typografie a publikov\'{a}n\'{i} \,--\, 4. projekt\\
\Huge Bibliografické citace
\vspace{\stretch{0.618}}
\end{center}
{\Large \today \hfill Petr Knetl}
\newpage
\pagenumbering{arabic}
\section*{Programovací jazyk Java}
Jazyk Java je v~dnešní době světový fenomén na~poli IT. O tom nás může přesvědčit výrok zkušeného programátora Pavla Herouta: \uv{\textit{Java je v dnešní době fenomén, který je skloňován ve~všech pádech. Za~pět let od~svého vzniku prošla neuvěřitelným rozvojem a stejně tak neuvěřitelně začíná ovlivňovat i~dění v~počítačovém světě}}~\cite{PHerJAVA}. Přes dnešní velkou působnost jazyku Java se jedná o poměrně nový jazyk. Počátky tohoto jazyka se vyskytují teprve v roce 1990, ve společnosti Sun Microsystems~\cite{JAVAhist}.

Mezi~nejsilnější stránky jazyku java patří jednoduchá přenositelnost mezi~zařízeními pomocí dnes hojně rozšířeného JVM neboli Java Virtual Machine~\cite{BogdanKiszka}. JVM je virtuální stroj, který poskytuje prostředí pro~programy jazyku Java, ve~kterém můžou bězet jejich procesy~\cite{JAVAjvm}. Nadruhou stranu, pokud se v~zařízení Java Virtual Machine nevyskytuje, tak jsou programy napsané v jazyku Java nepoužitelné. V~tomto směru jazyk Java zaostavá za~jinými programovacími jazyky, například jazykem~C~\cite{IanDarwin}.

Díky širokému rozšíření jazyku Java vznikla spousta užítečných knihoven s~předdefinovanými funkcemi~\cite{PHerJAVAlibs}. Další příjemnou vlastností jazyku je možnost naimplementovat grafické uživatelské rozhraní přímo v Javě~\cite{PHerJAVAgui}. Přesto že jde v~jazyku Java implementovat funkcionálním stylem programování, mezi~programátory převládá objektově orientovaný přístup~\cite{BruceEckel}.

Z~původního jazyku Java~se později vyvinul skriptovací jazyk JavaScript využívaný pro~implementaci interaktivity webových stránek~\cite{DaveThau}. Skripty napsané v~JavaScriptu přímo komunikují se soubory typu \texttt{.html}. Pomocí odkazů přímo v~souborech \texttt{.html} se přechází na~skripty JavaScriptu, které se po odkázání na~ně spustí~\cite{JSweb}.
\newpage
\bibliography{bibtex}

\end{document}
