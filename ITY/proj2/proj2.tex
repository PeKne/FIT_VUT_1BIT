\documentclass[a4paper, twocolumn, 11pt]{article}
\usepackage[left=1.5cm,top=2.5cm,text ={18cm,25cm}]{geometry}
\usepackage[czech]{babel}
\usepackage[utf8]{inputenc}
\usepackage{times}
\usepackage{amssymb}
\usepackage{amsmath}
\usepackage{amsthm}
\theoremstyle{definition}
\newtheorem{definition}{Definice}[section]

\begin{document}
\pagestyle{empty}
\onecolumn
\begin{center} 
\Huge
\textsc{Fakulta informačních technologií\\Vysoké učení
technické v Brně}\\
\vspace{\stretch{0.382}}

Typografie a publikování - 2. projekt\\
Sazba dokumentů a matematických výrazů
\vspace{\stretch{0.618}}
\end{center}
{\LARGE 2017 \hfill
Petr Knetl}
\setcounter{page}{0}
\newpage

\pagestyle{plain}
\twocolumn
\section*{Úvod} 
V této úloze si vyzkoušíme sazbu titulní strany, matematických vzorců, prostředí a dalších textových struktur obvyklých pro technicky zaměřené texty, například rovnice (1) nebo definice 1.1 na straně 1. 

Na titulní straně je využito sázení nadpisu podle optického středu s využitím zlatého řezu. Tento postup byl probírán na přednášce.
\section{Matematický text}
Nejprve se podíváme na sázení matematických symbolů a výrazů v plynulém textu. Pro množinu $V$ označuje card($V$) kardinalitu $V$
Pro množinu $V$ reprezentuje $V^{*}$ volný monoid generovaný množinou $V$ s operací konkatenace.
Prvek identity ve volném monoidu $V^{*}$ značíme symbolem $\varepsilon$
Nechť $V^{+} = V^{*} - \{\varepsilon\}$. Algebraicky je tedy $V^{+}$ volná pologrupa generovaná množinou $V$ s operací konkatenace.
Konečnou neprázdnou množinu $V$ nazvěme \textit{abeceda}.
Pro $\omega \in V^{*}$ označuje $|\omega|$ délku řetězce $\omega$. Pro $W \subseteq V$ označuje occur$({\omega}, W)$ počet výskytů symbolů z $W$ v řetězci $\omega$ a $sym({\omega}, i)$ určuje $i$-tý symbol řetězce $\omega$; například sym$(abcd, 3) = c$.

Nyní zkusíme sazbu definic a vět s využitím balíku \texttt{amsthm}.

\noindent \begin{definition} \textit{Bezkontextová gramatika} je čtveřice $G = (V, T, P, S)$, kde $V$ je totální abeceda,
$T \subseteq V$ je abeceda terminálů, $S \in (V - T)$ je startující symbol a $P$ je konečná množina pravidel
tvaru $q: A \rightarrow \alpha$, kde $A \in (V - T)$, $\alpha \in V^{*}$ a $q$ je návěští tohoto pravidla. Nechť $N = V - T$ značí abecedu neterminálů.
Pokud $q: A \rightarrow \alpha \in P$, ${\gamma}, \delta \in V^{*}$, $G$ provádí derivační krok z $\delta{A}\gamma$ do $\delta{\alpha}\gamma$ podle pravidla $q: A \rightarrow \alpha$, symbolicky píšeme 
$\delta{A}\gamma \Rightarrow \delta{\alpha}\gamma$ $[q: A \rightarrow \alpha]$ nebo zjednodušeně $\delta{A}\gamma \Rightarrow \delta{\alpha}\gamma$ . Standardním způsobem definujeme $\Rightarrow^{m}$, kde $m \geq 0$  . Dále definujeme 
tranzitivní uzávěr $ \Rightarrow^{+}$ a tranzitivně-reflexivní uzávěr $\Rightarrow^{*}$ .
\end{definition}

Algoritmus můžeme uvádět podobně jako definice textově, nebo využít pseudokódu vysázeného ve vhodném prostředí (například \texttt{algorithm2e}).

\noindent \textbf{Algoritmus 1.2.} Algoritmus pro ověření bezkontextovosti gramatiky. Mějme gramatiku $G = (N, T, P, S)$.
\textit{\begin{enumerate}
 \item Pro každé pravidlo $p \in P$ proveď test, zda $p$ na levé straně obsahuje právě jeden symbol z $N$.
 \item Pokud všechna pravidla splňují podmínku z kroku 1 tak je gramatika $G$ bezkontextová.
\end{enumerate} }


\noindent \textbf{Definice 1.3.} \textit{Jazyk} definovaný gramatikou $G$ definujeme jako $L(G) = \{ \omega \in T^{*}|S \Rightarrow^{*} \omega\}$.
\subsection{Podsekce obsahující větu}
\noindent \textbf{Definice 1.4.}: Nechť $L$ je libovolný jazyk. $L$ je \textit{bezkontextový jazyk}, když a jen když $L=L(G)$ kde $G$ je libovolná bezkontextová gramatika.

\noindent \textbf{Definice 1.5.}  Množinu ${\zeta}_{CF} = \{L|L$ je  bezkontextový jazyk\} nazýváme \textit{třídou bezkontextových jazyků}.

\noindent  \textbf{Věta 1.} : Nechť $L_{abc} = \{a^{n}b^{n}c^{n}|n \geq 0\}$. Platí, že $L_{abc} \not\in {\zeta}_{CF}$.

\noindent \textit{Důkaz}. Důkaz se provede pomocí Pumping lemma pro bezkontextové jazyky, kdy ukážeme, že není možné, aby platilo, což bude implikovat pravdivost věty  1. %odkaz

\section{Rovnice a odkazy}
Složitější matematické formulace sázíme mimo plynulý text. Lze umístit několik výrazů na jeden řádek, ale pak je třeba tyto vhodně oddělit, například příkazem \texttt{{\textbackslash}quad}. 

$$\sqrt[x^{2}]{y^{3}_0} \quad \mathbb{N} = \{0, 1 ,2,...\} \quad x^{y^{y}} \neq x^{yy} \quad z_{i_{j}} \not{\equiv} z_{ij}$$

V rovnici (...) jsou využity tři typy závorek s různou explicitně definovanou velikostí.
\setcounter{equation}{0}
\label{eq}\begin{equation}
\left\{ \left[(a + b) * c \right]^{d} +1 \right\} = x
\end{equation}
\begin{equation*}
\lim_{x\to \infty} \frac{sin^{2}x + cos^{2}x}{4} = y 
\end{equation*}
V této větě vidíme, jak vypadá implicitní vysázení limity $\lim_{n\to \infty}f(n)$ v normálním odstavci textu. Podobně je to i s dalšími symboly jako $\sum _{1}^n$ či ${\bigcup}_{A{\in}\beta}$ . V případě vzorce $\lim\limits_{x \to 0} \frac{sin x}{x} =1$ jsme si vynutili méně úspornou sazbu příkazem \texttt{{\textbackslash}limits} .

\begin{eqnarray}
\int_a^b\limits f(x) \, \mathrm{d}x. &=& -\int_b^a \! f(x) \, \mathrm{d}x. \\
(\sqrt[5]{x^4})' = (x^{\frac{4}{5}})' &=& \frac{4}{5}x^{-\frac{1}{5}} = \frac{4}{5\sqrt[5]{x}}\\
\overline{\overline{A \vee B}} &=& \overline{\overline{A}} \overline{\wedge} \overline{\overline{B}}
\end{eqnarray}

\section{Matice}

Pro sázení matic se velmi často používá prostředí array a závorky (\texttt{{\textbackslash}left}, \texttt{{\textbackslash}right}).


$$\left(
\begin{array}{c l}
a + b & b - a \\
\widehat{\varepsilon + \omega} & \enspace \hat{\pi} \\
\overrightarrow{a} & \overleftrightarrow{AC}\\
0 & \enspace \beta
\end{array} \right)
$$

$$A = \left|\left| {\begin{array}{cccc}
a_{11} & a_{12} & \ldots & a_{1n} \\
a_{21} & a_{22} & \ldots & a_{2n} \\
\vdots & \vdots & \ddots & \vdots \\
a_{m1} & a_{m2} & \ldots & a_{mn}
\end{array}} \right|\right| $$

$$\left| {\begin{array}{cc}
t & v \\
v & w
\end{array}}\right| = tw - uv $$


Prostředí \texttt{array} lze úspěšně využít i jinde.

$$\dbinom{n}{k} = \left\{
\begin{array}{c l}
\frac{n!}{k!(n-k)!} & \text{pro }  0 \leq k \leq n \\
0 & \text{pro } k < 0 \text{ nebo } k > n
\end{array} \right. $$

\section{Závěrem}

V případě, že budete potřebovat vyjádřit matematickou konstrukci nebo symbol a nebude se Vám dařit jej nalézt v samotném \LaTeX{u}, doporučuji prostudovat možnosti balíku maker \AmS{-}\LaTeX.
Analogická poučka platí obecně pro jakoukoli konstrukci v \TeX{u}.


\end{document}